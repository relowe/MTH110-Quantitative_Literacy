\documentclass{article}
\usepackage{fullpage}
\usepackage[T1]{fontenc}
\usepackage{titling}
\usepackage{graphicx}
\usepackage{enumitem}

\setlength{\droptitle}{-8em}   % This is your set screw
\title{Homework 1}
\date{}

\begin{document}
\maketitle

\section*{Reading Assignment}
Read the following before attempting the problems in this assignment.  
Your quiz next Wednesday will include questions to test whether you read and studied
these texts.
\begin{itemize}
    \item {\em Arithmetic for the Practical Man} Chapter I pages 3-7 \newline
    https://tinyurl.com/y88wt9et\newline
    \includegraphics[width=1in]{readings/practical1}
    
    \item {\em A Mathematical Solution Book} Chapter II pages 14-22 \newline
    https://tinyurl.com/y8qagw5w\newline
    \includegraphics[width=1in]{readings/solution1}
\end{itemize}

\section*{Appreciating Notation}
In the readings, we see that there were several competing systems of numbers used
in the ancient world.  The one most relevant to us in the present day is the Roman
numeral system because it is used when we want to be ``fancy''.  (The others are only
of interest during bar trivia nights or game shows.)  So then why did the Roman 
system fall into disuse for standard use?  Let's try out some arithmetic and 
see what happens! For the problems in this section, do not use a calculator.  
Try to invent some system of dealing with the Roman numerals. We talked about some in
class, but you'll need to do a little inventing here.  It's ok to get these wrong, 
I just want to see a reasonable attempt to work in this system.

\subsection*{What was Roman arithmetic like?}
Try your best to work complete in Roman numerals in this set of problems.
\begin{enumerate}
    \item $\mathtt{III} + \mathtt{V} = ?$
    \item $\mathtt{CCIV} + \mathtt{CLVII} = ?$
    \item $\mathtt{VII} - \mathtt{II} = ?$
    \item $\mathtt{C} - \mathtt{LI} = ?$
    \item $\mathtt{XXI} \times \mathtt{III} = ?$
    \item $\mathtt{CLXXXI} \times \mathtt{LXIX} = ?$
    \item If a person was born in the year $\mathtt{MCMLXXX}$, how old would they
      be in the year $\mathtt{MMXIX}$?
\end{enumerate}

\subsection*{Is the Arabic numeral system better?}
Now, let's try the problems from above, but this time in Arabic numerals.  Write 
these problems as you would have back when you were in grade school and solve them
by hand.  Isn't that so much nicer?
\begin{enumerate}[resume]
    \item $3+5=?$
    \item $204 + 258=?$
    \item $7 - 2 = ?$
    \item $100 - 51 = ?$
    \item $21 \times 3 = ?$
    \item $181 \times 69 = ?$
    \item If a person was born in the year 1980, how old would they be in the year 2019?
\end{enumerate}

\section*{Reasoning With Numbers and Notation}
\begin{enumerate}[resume]
    \item Write three hundred seventy-one quadrillion one
        hundred thirty-four trillion six hundred
        thirty-two billion two hundred ninety-eight million two thousand four hundred
        and nineteen as a number with proper use of commas and zeroes.
    \item Round the number from the previous problem to the nearest quadrillion and write it in scientific notation.
    \item Is a billion a million millions?  Explain your answer and demonstrate that it
      is true.
    \item Using proper order of operations, show all the reductions necessary to solve the following problems:
    \begin{enumerate}
        \item $12 \times 2 - 1$
        \item $13+4+5-1$
        \item $11-7\times 4 \div 2^3$
        \item $6 \div 2(1+2)$
    \end{enumerate}
\item Madison is a very happy student, for today she has started work in an amazing new
class called ``Quantitative Literacy''.  This course was designed by a team of geniuses
who want to help Madison avoid getting eaten alive by predatory lenders and corrupt political
parties. (Ok, let's face it, that last possibility is VERY wishful thinking.  We can at least save Madison's finances.) In this class, Madison's grade is weighted as follows:

    {\em Attendance} 15\%,  {\em Homework} 5\%, {\em Quizzes} 15\%,
    {\em Projects} 30\%, {\em Regular Exams} 15\%, {\em Comprehensive Final} 20\%

Madison has averaged her grades (and signed a FERPA release so I can share them with you.)  (Ok, she's actually fictional.  Any resemblance to any Madison living or dead is purely
coincidental.)  The averages in each category are:

    $Attendance=100, Homework=90, Quizzes=80, Projects=95, Exams=88, Final=84$
    
Write a single formula to calculate Madison's grade.  Show each reduction,
solving each operation of your expression until you compute Madison's final grade.
What is her numeric grade? What is her letter grade (according to the scale found on the syllabus)?  HINT: The syllabus has details of how to perform this calculation.  Reading
it is to your advantage.

\end{enumerate}


\end{document}

