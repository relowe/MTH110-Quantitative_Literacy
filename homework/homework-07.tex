\documentclass{article}
\title{Homework 7}
\date{October 7, 2019}
\usepackage[shortlabels]{enumitem}
\usepackage{graphicx}

\begin{document}
\maketitle

\section*{Reading Assignment}
\begin{center}
    {\em Math in Society} by David Lippman\newline
    \includegraphics[width=1in]{readings/society}\newline
    {\tt http://tinyurl.com/y54ndew6}
\end{center}
\begin{enumerate}
\item Math in Society pages 247-261 (pdf pages 251-265)
\item Math in Society pages 173-178 (pdf pages 177-182)
\end{enumerate}

\section{CPI Problems}
\begin{enumerate}
\item If a car cost \$700.00 in 1965, how much would that same car cost in 2018?
\item From 1970 to 2018, the United States has experienced three major recessions.  
Using the CPI chart, compute the rate of inflation over these 48 years.  When did
these recessions begin and end?  Support your reasoning using the CPI data.  (You 
may want to use a spreadsheet to speed up this process.)
\item Find a few (around three to five) articles pertaining to the recessions in the
previous question. When did economists say these recessions happened?  Does this agree
with your assessment from the CPI data?
\end{enumerate}

\section{Linear Growth Problems}
The following problems are from "Math in Society" page 193 (pdf page 197)
\begin{enumerate}[resume]
\item Marko currently has 20 tulips in his yard. Each year he plants 5 more.
    \begin{enumerate}[a.)]
    \item Write a recursive formula for the number of tulips Marko has
    \item Write an explicit formula for the number of tulips Marko has
    \end{enumerate}
\item Pam is a Disc Jockey. Every week she buys 3 new albums to keep her collection current. She currently owns 450 albums.
    \begin{enumerate}[a.)]
    \item Write a recursive formula for the number of albums Pam has
    \item Write an explicit formula for the number of albums Pam has
    \end{enumerate}
\item A store's sales (in thousands of dollars) grow according to the recursive rule $P_n=P_{n-1} + 15$, with initial population $P_0=40$.
    \begin{enumerate}[a.)]
    \item Calculate $P_1$ and $P_2$.
    \item Find an explicit formula for $P_n$.
    \item Use your formula to predict the store’s sales in 10 years
    \item When will the store’s sales exceed \$100,000?
    \end{enumerate}
\item The number of houses in a town has been growing according to the recursive rule $P_n=P_{n-1} + 30$, with initial population $P_0=200$.
    \begin{enumerate}[a.)]
    \item Calculate $P_1$ and $P_2$.
    \item Find an explicit formula for $P_n$
    \item Use your formula to predict the number of houses in 10 years.
    \item When will the number of houses reach 400 houses?
    \end{enumerate}
\item A population of beetles is growing according to a linear growth model. The initial population (week 0) was $P_0=3$, and the population after 8 weeks is $P_8=67$.
    \begin{enumerate}[a.)] 
    \item Find an explicit formula for the beetle population in week $n$ 
    \item After how many weeks will the beetle population reach 187?
    \end{enumerate}
\item The number of streetlights in a town is growing linearly. Four months ago ($n = 0$) there were 130 lights. Now ($n = 4$) there are 146 lights. If this trend continues,
    \begin{enumerate}[a.)]
    \item Find an explicit formula for the number of lights in month $n$
    \item How many months will it take to reach 200 lights?
    \end{enumerate}
\end{enumerate}

\end{document}
