\documentclass{article}
\title{Linear Growth}
\date{February 17, 2020}
\usepackage{amsmath}

\begin{document}
\maketitle
\section*{Explanatory and Response Variables}
\begin{itemize}
	\item Oftentimes, we study a system based on its inputs and its outputs.
	\item An {\bf explanatory} variable is a variable which explains some effect.
	\item A {\bf response} variable is a variable which quantifies a response.
	\item We typically want to model how a a response variable changes when the explanatory variable change.
	\item We use the predict future behavior of something under study.
\end{itemize}

\section*{Linear Patterns}
\begin{itemize}
	\item To explore and model data, we look for patterns.
	\item A linear pattern is one where the ratio of the change in the response variable to the change in the explanatory variable remains constant.
	\item Suppose we have response variable $y$ and explanatory variable $x$.  If we pick 
	two entries $(x_1, y_1)$ and $(x_2, y_2)$, then for all pairs of values, the ratio must remain the same:
	\[
	m = \dfrac{y_2 - y_1}{x_2-x_1}
	\]
	\item For example, are the following patterns linear?\newline
	\begin{tabular}{|r|r|}
		\hline
		$x$ & $y$ \\
		\hline
		1 & 20\\
		2 & 25\\
		4 & 35\\
		6 & 45\\
		\hline
	\end{tabular}
	\begin{tabular}{|r|r|}
		\hline
		$x$ & $y$ \\
		\hline
		1 & 20\\
		2 & 25\\
		4 & 39\\
		6 & 61\\
		\hline
	\end{tabular}
\end{itemize}

\section*{Recursive Relationship}
\begin{itemize}
	\item A linear pattern can be written as a recursive relationship.  The relationship is the sum of the previous value and then the constant change in the response variable. 
	\item This can be written as follows:
	\[
	P_n = P_{n-1} + (y_2 - y_1)
	\]
	\item Example:  Write the previous linear pattern as a recursive relationship.
	\item Recursive relationships work well for conceptualizing how the pattern changes from one value to the next, but it is not very good for projecting future values.  (Discuss: Why?)
\end{itemize}

\section*{Explicit Linear Functions}
\begin{itemize}
	\item A linear growth pattern can be characterized as a linear function.
	\item One of the most common ways to write a linear function is in {\bf slope-intercept} form.  This form looks like this:
	\[
	f(x) = \mathrm{m} x + \mathrm{b}
	\]
	\item Here, $\mathrm{m}$ is the constant ratio of change (the slope)
	\item $\mathrm{b}$ is the value of the function when $x=0$.  This is called the ``y-intercept'' because this is the value of the function when it crosses the $y$ access.
	\item Example: Find a linear function for the linear pattern in the previous example.
	\item Example: Graph this function for $x$ values 1 through 10.
\end{itemize}
\end{document}
