\documentclass{article}
\title{Lecture 8 - Units of Measurement}
\author{}
\date{February 4, 2019}
\usepackage{cancel}
\usepackage{amsmath}
\usepackage{fullpage}

\begin{document}
\maketitle

\section*{History of Units}
\begin{itemize}
\item Measurements of weight and volume were likely the first standardized units.  (People have always been obsessed with commerce and shiny rocks/metals!)
\item The ancient Greek geometers used ratio to characterize length.  They looked for some standardizing element that brought all sides of a shape into unity with one another.
\item Pythagoras incorrectly believed that every shape could be characterized by ratios of whole numbers.  When a student proved him wrong, the Pythagoreans drowned the wayward student in a well!
\item Standardized units of length have existed for at least 5,000 years, with one of the earliest being the cubit.
\item A modern carpenter's square actually bears a striking resemblance to the cubit square used in ancient Egypt!
\end{itemize}

\section*{Metric Units and Imperial Units}
\subsection*{The Metric System}
\begin{itemize}
\item All civilized nations, and a few barbarous ones, use the metric system.
\item The metric system uses the following base units:\newline
\begin{tabular}{|l|l|l|}
\hline
{\bf Type} & {\bf Name} & {\bf Abbreviation} \\
\hline
Length & meter & m \\
\hline
Mass & gram & g \\
\hline
Volume & liter & l \\
\hline
Temperature & Celsius/Kelvin & $^{\circ}$C / $^{\circ}$K\\
\hline
Time & second & s \\
\hline
\end{tabular}

\item Under the metric system, all unit conversions are done
by powers of ten.  Hence, all of the unit prefixes are really
just shorthand renderings of scientific notation! The powers of
ten and prefixes are shown below:\newline

\begin{tabular}{|l|l|c||l|l|c|}
\hline
{\bf Value} & {\bf Name} & {\bf Prefix} & {\bf Value} & {\bf Name} & {\bf Prefix} \\
\hline
$10^{-24}$ & yocta & y & $10^1$ & deka & da\\
\hline
$10^{-21}$ & zepto & z & $10^2$ & hecto & h\\
\hline
$10^{-18}$ & atto & a & $10^3$ & kilo & k\\
\hline
$10^{-15}$ & femto & f & $10^6$ & mega & M\\
\hline
$10^{-12}$ & pico & p & $10^9$ & giga & G\\
\hline
$10^{-9}$ & nano & n & $10^{12}$ & tera & T \\
\hline
$10^{-6}$ & micro & $\mu$ & $10^{15}$ & peta & p\\
\hline
$10^{-3}$ & milli & m & $10^{18}$ & exa & E\\
\hline
$10^{-2}$ & centi & c & $10^{21}$ & zetta & Z\\
\hline
$10^{-1}$ & deci & d & $10^{24}$ & yotta & Y\\
\hline
\end{tabular}

\end{itemize}

\subsection*{The Imperial System}
\begin{itemize}
\item The imperial system is an inconsistent hodgepodge of units 
currently only used by impoverished and despotic nations.
\item See the attached handout for a way to interpret this madness!
\item Conversion from the good and wise metric system to the wayward
imperial system works as follows:
\begin{itemize}
\item {\bf Length} 1 inch = 2.54 cm
\item {\bf Volume} 1 quart = 0.946353 l
\item {\bf Mass (sort of)} 1 pound = 0.453592 kg
\end{itemize}
\end{itemize}

\section*{Performing Unit Conversions}
\begin{itemize}
\item Units behave like numbers and/or variables in calculations.
\item Hence $2\mathrm{m} \times 3 \mathrm{m} = 6\mathrm{m}^2$
\item The most common way to convert units is to write them as a series of fractions and multiply.
This is called the ``factor label method''.  For example, converting 3km to miles works as follows
\newline
\[
3\mathrm{km} \times \dfrac{1000\mathrm{m}}{1\mathrm{km}} \times \dfrac{100\mathrm{cm}}{1\mathrm{m}} \times \dfrac{1\mathrm{in}}{2.54\mathrm{cm}} \times \dfrac{1 \mathrm{ft}}{12\mathrm{in}}\times \dfrac{1\mathrm{mile}}{5280\mathrm{ft}}
\]
\[
	3\cancel{\mathrm{km}} \times \dfrac{1000\cancel{\mathrm{m}}}{1\cancel{\mathrm{km}}} \times \dfrac{100\cancel{\mathrm{cm}}}{1\cancel{\mathrm{m}}} \times \dfrac{1\cancel{\mathrm{in}}}{2.54\cancel{\mathrm{cm}}} \times \dfrac{1 \cancel{\mathrm{ft}}}{12\cancel{\mathrm{in}}}\times \dfrac{1\mathrm{mile}}{5280\cancel{\mathrm{ft}}}
\]
\[
1.86\mathrm{miles}
\]

\subsection*{Examples}
\begin{enumerate}
\item What is the speed of sound in furlongs per fortnight?
\item How many liters are in one butt?
\end{enumerate}
\end{itemize}

\end{document}
