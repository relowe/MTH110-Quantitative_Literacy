\documentclass{article}
\usepackage{fullpage}
\usepackage{amsmath}
\usepackage{graphicx}
\usepackage[inline]{enumitem}

\title{Lecture 3 - Formulae and Functions}
\author{Robert Lowe}
\date{January 16, 2019}

\begin{document}
\maketitle

\section*{Quiz 1}
Good luck!

\section*{Formulae}
\begin{tabular}{|c|c|c|c|}
\hline
Area of a Circle & Circumference of a Circle & Area of a Rectangle & Perimeter of a Rectangle\\
$\pi r^2$ & $2\pi r$ & $l \times w$ & $2\times l + 2 \times w$\\
& & &\\
\hline
Area of a Triangle & Surface Area of a Sphere & Volume of a Sphere & Quadratic Formula\\
$\frac{1}{2}bh$ & $4 \pi r^2$ & $\frac{4}{3} \pi r ^3$ & $x = \displaystyle\frac{-b \pm \sqrt{b^2-4ac}}{2a}$\\
& & &\\
\hline
\end{tabular}\newline
$\pi=3.141592653589793238462643383279\ldots$

\begin{itemize}
\item A formula is a way of writing down a generic computation so it can be repeated 
 as many times as needed.
\item Letters serve as placeholders for numbers.  (We refer to these as variables.)
\item To apply a formula, we just fill in the numbers.
\end{itemize}

\subsection*{Examples}
\begin{enumerate}
\item What is the area of a circle which has a radius of $4\mathrm{cm}$? What is its circumference?
\item What is the perimeter of an American football field?  (A standard football field $53\frac{1}{3}\mathrm{yd}$ wide and $120 \mathrm{yd}$ long).  What is its area?
\item What is the surface area of a basketball? (The diameter of a basketball is $10\mathrm{in}$) What is its volume?
\end{enumerate}

\section*{Functions}
\begin{itemize}
\item A function is a rule which shows how one set maps onto another.  (Usually we mean one set of numbers onto another set of values.)
\item Algebraic functions are written like a formula\newline
\[
f(x) = x + 2
\]
This means that the function $f$ applies $x+2$ to the value it is given.  This allows a complete expression to be written using a formula.

\begin{align*}
f(2) &= 2 + 2 \\
f(2) &= 4
\end{align*}

Note that $f(x)$ is the notation that means ``Function of $x$'' and not $f \cdot x$.
\subsection*{Examples}
\begin{enumerate}
  \item Write each of the geometric formulae from the previous section as a function.
  \item In the first half of a basketball game, team A scored 60 points and team B scored 70 points.  In the second half, team A scores 7 points per minute and team B scores 8 points per minute.  Write two functions, one for team A and one for team B, representing each team's respective score in the second half.  If the second half of the game comprises 24 minutes of play, which team wins the game?
\end{enumerate}


\end{itemize}

\end{document}
