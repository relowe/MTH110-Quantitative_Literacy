\documentclass{article}
\title{Annuities an Loans}
\date{October 25, 2019}
\usepackage{amsmath}
\usepackage{mathtools}
\usepackage{textcomp}
\usepackage{multicol}
\usepackage{fullpage}

\begin{document}
\maketitle

%\begin{multicols*}{2}
\section{Annuities}
\begin{itemize}
    \item The compound interest formula assumes that we will deposit
        or borrow money and then never make payments or further deposits.
    \item Most of us cannot do this!
    \item A compound interest account into which we deposit regularly
        is called an \textbf{annuity}.
    \item Example annuities are savings accounts, retirement accounts
        (such as a 401k), and IRAs.
    \item Recall that the recursive formula for compound interest is:
        \[
            P_m = \left(1+\dfrac{r}{k}\right) P_{m-1}
        \]
    \item If, in addition to interest, we make a monthly deposit of
        $d$, this becomes:
        \[
            P_m = \left(1+\dfrac{r}{k}\right) P_{m-1} + d
        \]
    \item Suppose I deposit \$100.00 each month into a savings account
        which has 2.02\% APY (Annual Percentage Yield) compounded
        monthly.  Use the recursive formula to find the amount in the
        account over the first 3 months.
    \item Solving this recursive formula in the general case gives us
        the explicit annuity formula:
        \[
            P_N = \dfrac{d\left(\left(1+\dfrac{r}{k}\right)^{Nk} - 1 \right)}%
                        {\left(\dfrac{r}{k}\right)}
        \]
        Where:
        \begin{description}
            \item[$N$] - Number of Years
            \item[$k$] - Compounding Periods per Year
            \item[$d$] - Deposit per compounding period
            \item[$r$] - APY or APR
        \end{description}

    \item In my above scheme, how much will I have in my savings
        account after 10 years?
\end{itemize}

\section{Payout Annuities}
\begin{itemize}
    \item A payout annuity is one in which we make a regular monthly
        withdrawal of $d$.
    \item For instance, suppose we retire.  We stop paying into our
        retirement account, and it becomes a payout annuity.
    \item The formula for a payout annuity is as follows:
    \[
            P_0 = \dfrac{d\left(1-\left(1+\dfrac{r}{k}\right)^{-Nk} \right)}%
                        {\left(\dfrac{r}{k}\right)}
    \]
    Where $P_0$ is our starting balance.

    \item Suppose our retirement goal is to have a retirement income
        of \$50,000.00 per year.  Assuming our retirement account
        earns 6\% interest, we retire at 65, and we die at age 80, how
        much must we have in our account when we retire?

    \item Assuming we begin saving for retirement at age 22, how much
        must we save each month to reach our goal?
\end{itemize}

\section{Compound Interest Loans}
\begin{itemize}
    \item A compound interest loan is effectively a payout annuity, so
        it uses the same formula.
    \item The payout annuity formula also tells us the remaining
        balance in a loan, if we know the monthly payment.
    \item The deposits are your payment.  Solving the annuity formula
        for d, we can work out the payment for the loan:
        \[
        d = \dfrac{\left(\dfrac{r}{k}\right) P_0}%
                  {\left(1-\left(1+\dfrac{r}{k}\right)^{-Nk} \right)}
        \]
    \item The amortization schedule of a loan is the listing of how
        much interest is paid each payment period and how much
        principal remains.
    \item The full amortization of a loan tells us how much money
        will actually be repaid over the course of a loan.
    \item The difference between the value of the purchased asset
        and the remaining principal of the loan is called the
        \textbf{equity} of the asset.  Generally, only real-estate gains
        equity.
    \item Suppose we purchase a home at \$100,000.00 with a down
        payment of \$10,000.00 and finance the rest with a fixed-rate
        30 year mortgage at 5\% interest.  The loan compounds monthly.
        What will our monthly payment be?
    \item How much, in total, will we have paid for this house at the
        end of our loan period?
\end{itemize}
%\end{multicols*}


\end{document}
