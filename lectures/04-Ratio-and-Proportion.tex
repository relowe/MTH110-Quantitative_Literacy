\documentclass{article}
\usepackage{fullpage}
\usepackage{amsmath}
\usepackage{graphicx}
\usepackage[inline]{enumitem}

\title{Lecture 4 Ratio and Proportion}
\author{Robert Lowe}
\date{January 18, 2019}

\begin{document}
\maketitle
\section*{A Complex Sounding Problem}
If 8 workers in 24 days working 10 hours a day can reap 48 acres of wheat, how many acres could 12 workers reap in 20 days of 12 hours each?

\section*{Ratio}
\subsection*{Comparing and Relating Quantities}
\begin{itemize}
    \item A ratio compares quantities of like types. (days to days, dollars to dollars, 
    workers to workers, etc.)
    \item Ratios express the relationship between two concrete quantities.
\end{itemize}

\subsection*{Notation}
\begin{itemize}
    \item The notation for a ratio is $a:b$ Where $a$ is the first quantity and $b$ is the second.
    \item This is frequently pronounced as ``$a$ to $b$''
    \item A ratio is equiavelent to a fraction, thus $1:2$, $1\div 2$, $\frac{1}{2}$, and $0.5$ are all the same ratio.
\end{itemize}

\subsection*{Ratios, Categories, and Properties}
\begin{itemize}
    \item Often we use ratios to categorize populations of similar items.
    \item Example:  What is the ratio of women to men in this room? What is the ratio of men to women?
    \item Ratios can be reduced in the same way we reduce fractions.
    \item Ratios can always be compared, even when they represent ratios of disparate objects.
    \item Ratios are abstract numbers. Why?
\end{itemize}

\section*{Proportion}
\begin{itemize}
  \item A proportion is two ratios which represent the same fraction.  (Example: $1:2$ and $2:4$.)
  \item We often use the word ``in proportion'' to describe two equal ratios.
  \item Proportions are the mathematical equivalent of analogies: $a$ is to $b$ as $c$ is to $d$.
\end{itemize}
\subsection*{Proportion Notation and Properties}
\begin{itemize}
    \item There are two main ways to write proportions $a:b :: c:d$ or $a:b = c:d$ where $a$, $b$, $c$, and $d$ are the numbers which make up the proportion.  Example: $1:2::2:4$ or $1:2=2:4$.
    \item The two outer numbers are called the {\bf extremes} of the proportion.  $a:b::c:d$ has extremes $a$ and $d$.
    \item The two inner numbers are called the {\bf means} of the proportion.  $a:b::c:d$ has means $b$ and $c$.
    \item In order for four numbers to be in proportion, the product of the extremes must equal the product of the means.  So in $a:b::c:d$, $a\times d = b \times c$.
    \item Examples: $1:2::2:4$, $1:3::6:18$
    \item Discuss: Why must these two products be the same?
\end{itemize}

\subsection*{The Rule of Three}
\begin{itemize}
    \item If three numbers of a proportion are known, the fourth may be found.  
    \item Missing Mean (if we know $a$, $c$, and $d$)
    \begin{align*}
        a:x &:: c:d\\
        x\cdot c &= a \cdot d\\
        x &=\dfrac{a\cdot d}{c}
    \end{align*}
    
    \item Missing Extreme (if we know $b$, $c$, $d$)
    \begin{align*}
        x:b &:: c:d\\
        x\cdot d &= b \cdot c \\
        x &= \dfrac{b \cdot c}{d}
    \end{align*}
\end{itemize}

\subsubsection*{Example Problems}
\begin{enumerate}
\item Assuming that all classes maintain the same proportion of men and women as this one,
if a class had 20 men, how many women would it have?
\item A besieged town, containing 22,400 inhabitants, has provisions to
last 3 weeks; how many must be sent away that they may be able to hold out 7 weeks? \footnotesize{Transcribed from: {\em A Treatise on Arithmetic} by J. H. Smith. 1878}
\end{enumerate}

\subsection*{Compound Proportions}
\begin{itemize}
    \item A compound proportion is a set of three or more ratios given where one is incomplete.
    \item You produce solutions to compound proportions by multiplying corresponding terms together, and then solve 
    as in a simple proportion.
    \item Our opening example is a compound proportion.
\end{itemize}
\end{document}
