\documentclass{article}
\title{Debt and Simple Interest}
\date{March 2, 2020}
\usepackage{amsmath}
\usepackage{mathtools}

\begin{document}
\maketitle

\section*{Lending and Borrowing}
\begin{itemize}
    \item The \textbf{Principal} of a loan is the amount of money
        which is borrowed.
    \item The \textbf{Interest} of a loan is an additional amount
        which is paid by the lender.
    \item A lender makes profit by charge interest.
    \item Different loans have different terms of repayment.
\end{itemize}

\section*{Simple One-Time Interest}
\begin{itemize}
    \item $P_0$ is the principal of the loan.
    \item One type of loan is \textbf{Simple One-Time Interest}.  This
        is where the interest is a flat rate applied to the principal.
    \[
        I=P_0r
    \]
    Where $r$ is the rate of interest on the loan.

    \item The total amount repaid is
    \[
        A=P_0 + I = P_0 + P_0r = P_0(1+r)
    \]
    Note that in the above, we write the rate as decimal.

    \item This type of loan is typically an agreement between friends.  

    \item For example: Suppose a friend loans you \$300.00 with an
    agreement that you will repay your friend a one time 5\% interest
    rate.  What is the total amount that you pay?
\end{itemize}

\section*{Simple Interest Over Time}
\begin{itemize}
    \item Another type of simple interest is when the interest is
        re-applied over time.
    \item One example of this is a bond which pays interest every year
        until it matures.
    \item Here the interest is:
    \[
    I = P_0rt
    \]

    \item The total value of the loan is:
    \[
        A = P_0 + I = P_0+P+0rt = P_0(1+rt)
    \]

    \item $t$ is the number of time periods for which the loan is held.

    \item Example: You invest \$1000.00 in a savings bond that yields
    an annual percentage rate (APR) of 5\% interest and matures after
    10 years.  How much total money will  you have at the end of 10
    years?
\end{itemize}
\end{document}
