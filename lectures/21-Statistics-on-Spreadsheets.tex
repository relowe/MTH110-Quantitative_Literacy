\documentclass{article}
\title{Statistics on Spreadsheets and Calculators}
\date{November 18, 2019}
\usepackage{amsmath}
\usepackage{mathtools}
\usepackage{textcomp}
\usepackage{multicol}
\usepackage{fullpage}
\usepackage{graphicx}

\begin{document}
\maketitle

\section{Spreadsheets}
\begin{itemize}
\item Spreadsheets provide a fast way to compute statistics.
\item We typically will enter our cases as a column, maybe with some
    sort of column heading.
\item The general ``language'' of spreadsheets is that each row is
    a case and each column is a variable within that case.
\end{itemize}


\subsection{Calculating the Five Number Summary}
\begin{itemize}
\item Label rows beneath the data with the five number summary
    variables: $min$, $Q_1$, $median$, $Q_3$, $max$
\item Each of these can be calculated with a spreadsheet function:
    \begin{itemize}
    \item \textbf{$min$}  \verb!=min(range)!
    \item \textbf{$Q_1$}  \verb!=quartile(range, 1)!
    \item \textbf{$median$}  \verb!=median(range)!
    \item \textbf{$Q_3$}  \verb!=quartile(range, 3)!
    \item \textbf{$max$}  \verb!=max(range)!
    \end{itemize}
\end{itemize}

\subsection{Calculating Mean and Standard Deviation}
\begin{itemize}
\item Label rows beneath the data with $mean$ and $stdev$.
\item Both are calculated using a functions:
    \begin{itemize}
    \item \textbf{Mean}  \verb!=average(range)!
    \item \textbf{Standard Deviation}  \verb!=stdev(range)!
    \end{itemize}
\end{itemize}

\section{Calculator Statistics}
Most scientific calculators can compute statistics as well.

\subsection{TI-83/84 Plus}
\begin{enumerate}
    \item Press [STAT]
    \item Select \verb!Edit...!
    \item Enter your data in one of the lists. 
    \item Press [STAT]
    \item Select \verb!CALC! from the top.
    \item Select \verb!1-Var Stats! and press enter
    \item Press [2nd] and the corresponding \verb!L! for your list.
        For instance, if you used \verb!L1! press [2nd][1]
\end{enumerate}
\begin{itemize}
    \item Press enter and your calculator will display all the
        statistics.
    \item Note that you can use multiple lists, and that you can clear
        the lists from the the [STAT] mode.
\end{itemize}

\subsection{TI-30X II}
\begin{enumerate}
    \item Press [2nd] [DATA] 
    \item Select \verb!1-VAR! to enter Statistics Mode
    \item Press [DATA]
    \item Enter your values as follows: 
        \begin{enumerate}
            \item Enter your value in $X_1=$.
            \item Press down arrow.
            \item Enter the number of times this value occurs in
                $FRQ=$.
            \item Only enter unique values.
        \end{enumerate}
    \item Press [STATVAR] and scroll through the statistics it
        generates:
        \begin{description}
            \item[$n$] -- Number of cases
            \item[$\bar{x}$] -- Mean
            \item[$Sx$] -- Sample Standard Deviation
            \item[$\sigma x$] -- Population Standard Deviation
            \item[$\Sigma x$] -- Sum of all cases
            \item[$\Sigma x^2$] -- Sum of the square of all cases
        \end{description}
    \item Press [2nd][DATA] right right enter to clear data.
    \item Press [2nd][STATVAR] to exit statistics mode.
\end{enumerate}

\subsection{Casio Calculators}
\begin{enumerate}
    \item [Mode][.] to enter statistics mode (may vary by model)
    \item Enter each case, pressing [M+] (or [$\Sigma+$] after each
        entry.
    \item Use [Shift] to see variables.  Look for $\bar{x}, \sigma n,
    \sigma n-1, \Sigma x^2, \Sigma x,$ and $n$.  These are the mean,
        population standard deviation, sample standard deviation, sum,
        sum of squares, and number of cases respectively.
\end{enumerate}
\end{document}

