\documentclass{article}
\title{Graphing and Projection Using Linear Functions}
\date{February 10, 2020}
\usepackage{amsmath}
\usepackage{mathtools}

\begin{document}
\maketitle

\section*{Linear Functions}
\begin{itemize}
    \item Recall that a linear pattern can be written recursively:
	\[
	P_n = P_{n-1} + (P_2 - P_1)
	\]

    \item Suppose we carry this out for some arbitrary number of iterations:
    \begin{align*}
    P_x &= P_{x-1} + (P_2 - P_1) \\
    P_x &= (P_{x-2} + (P_2 - P_1)) + (P_2 - P_1) \\
    P_x &= ((P_{x-3} + (P_2 - P_1)) + (P_2 - P_1)) + (P_2 - P_1) \\
    \ldots &\\
    P_x &= ((P_1 + \ldots + (P_2 - P_1)) + (P_2 - P_1)) + (P_2 - P_1) \\
    \end{align*} 

    \item Regrouping by using the associative property of addition, we get:
    \[
    P_x = P_1 + (P_2 - P_1) + \ldots + (P_2 - P_1)
    \]
    Where $(P_2 - P_1)$ is repeated $x$ times.  

    \item By the definition of multiplication, this gives us :
    \[
    P_x = P_1 + x(P_2 - P_1)
    \]

    \item Moreover, because each iteration increases $n$ by 1, we can make the
    claim that $P_2 - P_1$ is the slope because:
    \[
        m = \displaystyle\frac{y_2 - y_1} {x_2 - x_1}
    \]

    \item This gives us:
    \[
    P_x = P_1 + xm
    \]

    \item If we want to standardize the starting point to be where $x=0$ on
        a cartesian plane, we can do this as:
        \[
        P_x = b + xm
        \]
        Where $b=P_0$.
    \item This gives us the familiar equation of a line:
        \[
        \boxed{y = mx + b}
        \]

    \item Of course, $b$ is rarely given to us, so we must solve for this:
        \begin{align*}
            y &= mx + b\\
            y - b &= mx \\
            -b &= mx - y \\
            b &= -mx + y \\
        \end{align*}
        Therefore:
        \[
            \boxed{b = y - mx}
        \]
    \item We find $b$ by substituting an arbitrary $(x,y)$ pair into the above.
    \item Thus we can write a linear pattern as a function:
    \[
        \boxed{f(x) = mx + b}
    \]
\end{itemize}

\section*{Graphing Linear Funcions}
\begin{itemize}
    \item To graph a linear function, simply choose two points and draw them on
        the graph.
    \item Draw a straight line through both points.
    \item This will project the linear function backwards and forewards.
    \item We can also make predictions based on the numeric value of the linear
        function.
    \item Also, we can use a spreadsheet to find the equation of a line from
        a data series by using the ``trendline'' feature.
\end{itemize}

\end{document}
