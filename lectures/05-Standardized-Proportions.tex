\documentclass{article}
\title{Lecture 5  Standardized Proportions and the Number 100}
\author{Robert Lowe}
\date{January 17, 2020}
\usepackage{fullpage}
\usepackage{amsmath}

\begin{document}
\maketitle

\section*{Percentage Terms and Notation}
\[
\mathrm{percentage} : \mathrm{base} :: \mathrm{rate} : 100
\]
\subsection*{Terms}
\begin{itemize}
    \item Percent is a standardize proportion where a ratio between percentage and base is related to parts of 100. (Literally the same as saying ``$x$ out of 100'')
    \item The {\bf percentage} is the part of a number computed by the rate.
    \item The {\bf base} is the number on which the percentage is computed.  (This can often be thought of as the total amount, original amount, or total population in most problems.)
    \item The {\bf rate}, also referred to as the percent, is the parts out of 100 to be taken from the base. 
    \item {\bf Amount} is the sum obtained by adding the percentage to the base.
    \item {\bf Difference} is the remainder obtained by subtracting the percentage from the base.
\end{itemize}


\subsection*{Notation}
\begin{itemize}
\item A percent may be written as:
\begin{itemize}
  \item A ratio $25 : 100$ or $1:4$.
  \item A fraction $\dfrac{25}{100}$ or $\dfrac{1}{4}$
  \item A decimal $0.25$
  \item Using the $\%$ sign $25\%$
\end{itemize}
\item Frequently, a problem can be searched for keywords.  For example: ``What {\bf is} $25\%$ {\bf of} 200?''.
\begin{itemize}
  \item The ``is'' portion corresponds to the percentage.
  \item The ``of'' portion corresponds to the base.
  \item We could rewrite the fraction's proportion as the following mnemonic
  \[
  \mathrm{is} : \mathrm{of} :: \mathrm{percent} : 100
  \]
  \item Exercise: Rewrite this mnemonic proportion in fraction form.
\end{itemize}
\end{itemize}

\section*{Percentage Calculations}
\begin{itemize}
    \item To find any part of a percent, simply set up the proportion and solve.
    \begin{enumerate}
    \item What is $25\%$ of $300$?
    \item $120$ is $30\%$ of what number?
    \item What percent of $400$ is $50$?
    \end{enumerate}
    \item In problems dealing with amounts and differences, the base and percentage are used in the sum or difference.
    \begin{enumerate}
        \item A store sells shirts for $\$15.00$ apiece. If they have a $20\%$ off sale, what is the price of the shirts?
        \item A merchant purchases rugs for $\$10.00$ apiece and sells them for $\$15.00$.  What percent markup has the merchant applied?
        \item According to {\tt worldometers.info}, the United States population increases by $0.71\%$ each year.  If the present population of the United States is $3.28 \times 10^{8}$ people, what will the population be next year?
    \end{enumerate}
\end{itemize}

\section*{Percentage and Proportion Problems}
\begin{enumerate}
  \item A besieged town, containing 22,400 inhabitants, has provisions to last 3 weeks; how many must be sent away that they may be able to hold out 7 weeks? \footnote{Transcribed from: {\em A Treatise on Arithmetic} by J. H. Smith. 1878 \label{fn:treatise}}
  \item  If 8 workers in 24 days working 10 hours a day can reap 48 acres of wheat, how many acres could 12 workers reap in 20 days of 12 hours each?
  \item If a staff of $4\mathrm{ft}$ casts a shadow $7\mathrm{ft}$ in length, what is the height of a tower which casts a shadow of $198\mathrm{ft}$ at the same time? \footnote{Transcribed from: {\em The Proressive Higher Arithmetic} edited by Daniel Fish. 1878\label{fn:higher}}
  \item A homeowner sells their house at a loss of $20\%$.  If the selling price was 
  $\$60,000.00$, what was the original price of the home?
  \item In the erection of a house I paid twice as much for material
  as for labor.  Had I paid $6\%$ more for material, and $9\%$ more
  for labor, my house would have cost $\$1284.00$; what was its
  cost?\textsuperscript{\ref{fn:higher}}.
\end{enumerate}
\end{document}
