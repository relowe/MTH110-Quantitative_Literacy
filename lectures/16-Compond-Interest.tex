\documentclass{article}
\title{Compound Interest}
\date{October 23, 2019}
\usepackage{amsmath}
\usepackage{mathtools}
\usepackage{textcomp}

\begin{document}
\maketitle
\section{Compound Interest}
\begin{itemize}
    \item Most investments and loans use compound interest.
    \item In compound interest, interest is added to the principal
        of the loan after each compounding period.
    \item Example:  Suppose you borrow \$100.00 at 5\% interest
        compounded annually. If you pay nothing on this loan, how much
        will you owe after 5 years?
    \item What kind of growth is this?
    \item What is the effective interest rate over the life of the
        loan?
\end{itemize}

\section{Compound Interest Formula}
\begin{itemize}
    \item A compound interest transaction always has three parameters:
    \begin{enumerate} 
        \item $P_0$ - The principal amount of the transaction.
        \item $r$ - The annual percentage rate (APR) of the
            transaction.
        \item $k$ - The number of compounding periods per year
    \end{enumerate}

    \item To find a formula for the value of a compound interest
        transaction for a given number of compounding periods $t$ we 
        can apply the exponential growth formula:
        \[
            P_n = P_0(1+r)^n
        \]
    \item Typically, we view the life a loan in years and so we want
        to adjust this to allow us to calculate years.  Really, what
        the above computes is compound interest if we compound exactly
        once per year.  
    \item If we compound $k$ times per year, the effective rate of
        each period becomes $\frac{r}{k}$.
    \item If we compound $k$ periods over $N$ years, we have $Nk$
        compounding periods.
    \item Putting this all together gives us the yearly compound
        interest formula:
        \[
            P_N = P_0\left(1+\displaystyle\frac{r}{k}\right)^{Nk}
        \]
        Where the variables are:
        \begin{itemize}
            \item $P_0$ - Principal
            \item $r$ - Annual Percentage Rate
            \item $k$ - The Number of Compound Periods per Year
            \item $N$ - The Number of Years
        \end{itemize}
    \item Let's go online and shop for credit cards.  Each row of the
        class will pick a card and find its terms.  Then let's compute how
        much the balance on this card would be after the purchase of
        a Nintendo\texttrademark Switch and make no payments for 2 years.
\end{itemize}


\end{document}
