\documentclass{article}
\title{Exponential Functions}
\date{March 4, 2020}
\usepackage{amsmath}
\usepackage{mathtools}

\begin{document}
\maketitle

\section{Exponential Patterns}
\begin{itemize} 
    \item Not all growth or decay models are linear.
    \item Consider the following pattern

	\begin{tabular}{|r|r|}
		\hline
		$x$ & $y$ \\
		\hline
		1 & 10\\
		2 & 11\\
		3 & 12.1\\
        4 & 13.31\\
        5 & 14.641\\
		\hline
	\end{tabular}

    \item Is this pattern linear?
    \item How can we characterize this pattern?
\end{itemize}

\section{Formulae for Exponential Pattern}
\begin{itemize}
    \item A pattern is exponential if the response variable increases
        by the same rate each time the explanatory variable increases
        by 1.
    \item Recursive Formula:
    \[
        P_n = (1+r)P_{n-1}
    \]
    \item Explicit Formula:
    \[
        P_n = (1+r)^nP_0
    \]
    \item In both of the above, $r$ is a percent written as a decimal.
    \item What is the recursive formula for the pattern in the
        previous section?
    \item What is the explicit formula for the pattern in the previous
        section?
    \item Graph this pattern.  What is its shape like?
\end{itemize}

\section{Sample Problems}
\begin{enumerate}
    \item The current population of box turtles in the Maryville
        College Woods is approximately 300 turtles.  Assuming that the
        population increases by 5\% every year, create a chart showing
        the turtle population over the next 10 years.
    \item Draw a graph of the turtle population.
    \item Amoeba are single celled organisms which reproduce by
        mitosis.  An amoeba undergoes mitosis every 30 minutes.  That
        is, it divides into two amoeba.  If you have a culture that
        contains 100 Amoeba, how many will it contain in 24 hours
        (assuming none die)?
\end{enumerate}
\end{document}
