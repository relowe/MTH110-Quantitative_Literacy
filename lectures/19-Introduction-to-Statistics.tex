\documentclass{article}
\title{Probabilities and Games}
\date{November 11, 2019}
\usepackage{amsmath}
\usepackage{mathtools}
\usepackage{textcomp}
\usepackage{multicol}
\usepackage{fullpage}

\begin{document}
\maketitle

\section{Describing Data}
\begin{itemize}
    \item \textbf{Statistics} are metrics which are used to summarize data.
    \item Statistics can also be used to make predictions about data.
    \item When describing data, what we are interested in is:
    \begin{itemize}
        \item How are data distributed?
        \item What is the center of the distribution?
        \item What does a typical example look like?
        \item How spread out are the data?
        \item How do different distributions of data compare to each other?
    \end{itemize}
    \item For example, let's suppose you want to explore the Exam 2 scores of the 
        Spring 2019 and Fall 2019 sections of Quantitative Literacy.
        \begin{itemize}
            \item \textbf{Spring 2019 Scores}
            \newline\begin{tabular}{rrrrrrrrrr}
            18 & 27 & 27 & 34 & 36 & 36 & 43 & 50 & 57 & 59\\
            59 & 64 & 64 & 68 & 73 & 73 & 73 & 75 & 80 & 82\\
            84 & 100\\
            \end{tabular}

            \item \textbf{Fall 2019 Scores}
            \newline\begin{tabular}{rrrrrrrrrr}
            57 & 58 & 62 & 68 & 69 & 74 & 83 & 85 & 88 & 89 \\
            93 & 93 & 94 & 97 & 97 & 99 & 100 & 102 & 103 & 106\\
            106 \\
            \end{tabular}
        \end{itemize}
    \item Which section did better?
    \item How can we quantify who did better?
\end{itemize}

\section{Describing data Graphically}
\begin{itemize}
    \item One easy way to summarize data is graphically.
    \item \textbf{Stem and Leaf Plots} provide a quick way to summarize data.
    \item To construct a stem and leaf plot:
    \begin{enumerate}
        \item Look at the range of values.
        \item Write down ``stems'' this is the set of all but the last digits of the data.
           (for example, in the above data, the stems are the numbers 1-10)
        \item Beside each stem, write down the leaves (the last digits) of the numbers.
        \item Sort the leaves from smallest to largest.
    \end{enumerate}
    \item Let's make a stem and leaf plot of the Spring and Fall scores.
    \item What does this tell us?
    \item If we put the stem and leaf plots back to back, we can compare the distributions of the two.
    \item A \textbf{histogram} is a bar chart of frequencies.
    \item To construct a histogram:
    \begin{enumerate}
        \item Determine the range of ``bins'' we are going to have.
        \item Count the number of data elements that fall within each bin.
        \item Draw a bar chart of these frequencies.
    \end{enumerate}
    \item Let's construct a histogram of the Spring and Fall exam data.
    \item We can compare histograms by placing them back to back.
    \item What does this tell us?  Where is the center?  What is the distribution of grades like?
\end{itemize}
\end{document}
