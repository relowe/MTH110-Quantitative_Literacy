\documentclass{beamer}
\mode<presentation>
{
  \usetheme{Warsaw}
  \definecolor{mcgarnet}{rgb}{0.38, 0, 0.08}
  \definecolor{mcgray}{rgb}{0.6, 0.6, 0.6}
  \setbeamercolor{structure}{fg=mcgarnet,bg=mcgray}
  %\setbeamercovered{transparent}
}


\usepackage[english]{babel}
\usepackage[latin1]{inputenc}
\usepackage{times}
\usepackage[T1]{fontenc}
\usepackage{tikz}
\usepackage{graphicx}
\usepackage{adjustbox}
\usepackage{xcolor}

\newcommand{\imagesource}[1]{{\centering\hfill\break\hbox{\scriptsize Image Source:\thinspace{\small\itshape #1}}\par}}

\title{03 - Formulae and Functions}


\author{Dr. Robert Lowe\\}

\institute[Maryville College] % (optional, but mostly needed)
{
  Division of Mathematics and Computer Science\\
  Maryville College
}

\date[]{}
\subject{}

\pgfdeclareimage[height=0.5cm]{university-logo}{images/Maryville-College}
\logo{\pgfuseimage{university-logo}}



\AtBeginSection[]
{
  \begin{frame}<beamer>{Outline}
    \tableofcontents[currentsection]
  \end{frame}
}


\begin{document}

\begin{frame}
  \titlepage
\end{frame}

\begin{frame}{Outline}
  \tableofcontents
\end{frame}


% Structuring a talk is a difficult task and the following structure
% may not be suitable. Here are some rules that apply for this
% solution: 

% - Exactly two or three sections (other than the summary).
% - At *most* three subsections per section.
% - Talk about 30s to 2min per frame. So there should be between about
%   15 and 30 frames, all told.

% - A conference audience is likely to know very little of what you
%   are going to talk about. So *simplify*!
% - In a 20min talk, getting the main ideas across is hard
%   enough. Leave out details, even if it means being less precise than
%   you think necessary.
% - If you omit details that are vital to the proof/implementation,
%   just say so once. Everybody will be happy with that.

\section{Formulae}

\begin{frame}{Formulae}
\begin{adjustbox}{max width=0.9\textwidth}
    \begin{tabular}{|c|c|c|c|}
    \hline
    Area of a Circle & Circumference of a Circle & Area of a Rectangle & Perimeter of a Rectangle\\
    $\pi r^2$ & $2\pi r$ & $l \times w$ & $2\times l + 2 \times w$\\
    & & &\\
    \hline
    Area of a Triangle & Surface Area of a Sphere & Volume of a Sphere & Quadratic Formula\\
    $\frac{1}{2}bh$ & $4 \pi r^2$ & $\frac{4}{3} \pi r ^3$ & %
       $x = \displaystyle\frac{-b \pm \sqrt{b^2-4ac}}{2a}$\\
    & & &\\
    \hline
    \end{tabular}
\end{adjustbox}
$\pi=3.141592653589793238462643383279\ldots$

\begin{itemize}[<+->]
\item A formula is a way of writing down a generic computation so it
    can be repeated as many times as needed.
\item Letters serve as placeholders for numbers.  (We refer to these
    as variables.)
\item To apply a formula, we just fill in the numbers.
\end{itemize}
\end{frame}

\begin{frame}{Examples}
    What is the area of a circle which has a radius of
        $4\mathrm{cm}$? What is its circumference?
        \begin{enumerate}[<+(1)->]
            \item \textcolor{green}{$A=\pi r^2$}
            \item \textcolor{green}{$A=3.14 \times (4\mathrm{cm})^2$}
            \item \textcolor{green}{$A=3.14 \times 16\mathrm{cm^2}$}
            \item \textcolor{green}{$A=50.24\mathrm{cm^2}$}
            \item \textcolor{green}{$C=2\pi r$}
            \item \textcolor{green}{$C=2 \times 3.14 \times 4\mathrm{cm}$}
            \item \textcolor{green}{$C=6.28 \times 4\mathrm{cm}$}
            \item \textcolor{green}{$C=25.12\mathrm{cm}$}
        \end{enumerate}
\end{frame}

\begin{frame}{Examples}
    What is the perimeter of an American football field?  (A
        standard football field $53\frac{1}{3}\mathrm{yd}$ wide and $120
        \mathrm{yd}$ long).  What is its area?
    \begin{enumerate}[<+(1)->]
        \item \textcolor{green}{$P=2\times l + 2\times w$}
        \item \textcolor{green}{$P=2 \times 120\mathrm{yd} + 2 \times 53\frac{1}{3}\mathrm{yd}$}
        \item \textcolor{green}{$P=240\mathrm{yd} + 2 \times 53\frac{1}{3}\mathrm{yd}$}
        \item \textcolor{green}{$P=240\mathrm{yd} + 106\frac{2}{3}\mathrm{yd}$}
        \item \textcolor{green}{$P=346\frac{2}{3}\mathrm{yd}$}
        \item \textcolor{green}{$A=l \times w$}
        \item \textcolor{green}{$A=120\mathrm{yd} \times 53\frac{1}{3}\mathrm{yd}$}
        \item \textcolor{green}{$A=6,400\mathrm{yd^2}$}
    \end{enumerate}
\end{frame}

\begin{frame}
    What is the surface area of a basketball? (The diameter of
        a basketball is $10\mathrm{in}$) What is its volume?
    \begin{enumerate}[<+(1)->]
        \item \textcolor{green}{$S=4\pi r^2$}
        \item \textcolor{green}{$S=4 \times 3.14 \times (5\mathrm{in})^2$}
        \item \textcolor{green}{$S=4 \times 3.14 \times 25\mathrm{in^2}$}
        \item \textcolor{green}{$S=12.56 \times 25\mathrm{in^2}$}
        \item \textcolor{green}{$S=314\mathrm{in^2}$}
        \item \textcolor{green}{$A=\frac{4}{3} \pi r^3$}
        \item \textcolor{green}{$A=\frac{4}{3} \times 3.14 \times (5\mathrm{in})^3$}
        \item \textcolor{green}{$A=\frac{4}{3} \times 3.14 \times 125\mathrm{in^3}$}
        \item \textcolor{green}{$A\approx 1.33 \times 3.14 \times 125\mathrm{in^3}$}
        \item \textcolor{green}{$A\approx 4.18 \times 125\mathrm{in^3}$}
        \item \textcolor{green}{$A\approx 522.5\mathrm{in^3}$}
    \end{enumerate}
\end{frame}

\section{Functions}
\begin{frame}{Functions}
\begin{itemize}[<+->]
    \item A function is a rule which shows how one set maps onto
        another.  (Usually we mean one set of numbers onto another set of
        values.)
    \item Algebraic functions are written like a formula\newline
    \[
        f(x) = x + 2
    \]
    \item $f$ applies $x+2$ to the given value.  For example:
        \begin{align*}
        f(2) &= 2 + 2 \\
        f(2) &= 4
        \end{align*}
    \item Note that $f(x)$ is the notation that means ``Function of $x$'' and not $f \cdot x$.
    \end{itemize}
\end{frame}

\begin{frame}{Examples}
  Write each of the geometric formulae from the previous section as a function.

  \begin{columns}
  \column{0.5\textwidth}
  Area of a Circle
  \begin{enumerate}[<+(1)->]
    \item \textcolor{green}{$\pi r^2$}
    \item \textcolor{green}{$A(r) = \pi r^2$}
  \end{enumerate}

  Circumference of a Circle
  \begin{enumerate}[<+(1)->]
    \item \textcolor{green}{$2\pi r$}
    \item \textcolor{green}{$c(r) = 2\pi r$}
  \end{enumerate}

  Area of a Rectangle
  \begin{enumerate}[<+(1)->]
    \item \textcolor{green}{$l \times w$}
    \item \textcolor{green}{$A(l, w) = l \times w$}
  \end{enumerate}

  Perimeter of a Rectangle
  \begin{enumerate}[<+(1)->]
    \item \textcolor{green}{$P(l, w) = 2 \times l + 2 \times w$}
  \end{enumerate}

  \column{0.5\textwidth}
  Area of a Triangle
  \begin{enumerate}[<+(1)->]
    \item \textcolor{green}{$\frac{1}{2} b h$}
    \item \textcolor{green}{$A(b,h) = \frac{1}{2} bh$}
  \end{enumerate}

  Surface Area of A Sphere
  \begin{enumerate}[<+(1)->]
    \item \textcolor{green}{$4 \pi r^2$}
    \item \textcolor{green}{$S(r) = 4 \pi r^2$}
  \end{enumerate}

  Volume of a Sphere
  \begin{enumerate}[<+(1)->]
    \item \textcolor{green}{$\frac{4}{3} \pi r^3$}
    \item \textcolor{green}{$V(r) = \frac{4}{3} \pi r^3$}
  \end{enumerate}
  \end{columns}
\end{frame}

\begin{frame}{Examples}
  In the first half of a basketball game, team A scored 60 points and
  team B scored 70 points.  In the second half, team A scores 7 points
  per minute and team B scores 8 points per minute.  
  \newline\newline 
  Write two functions, one for team A and one for team B, representing each
  team's respective score in the second half.  If the second half of
  the game comprises 24 minutes of play, which team wins the game?

  \begin{columns}
  \column{0.5\textwidth}
  \begin{enumerate}[<+(1)->]
    \item \textcolor{green}{$s_a(x) = 60 + 7x$}
    \item \textcolor{green}{$s_a(24) = 60 + 7\cdot 24$}
    \item \textcolor{green}{$s_a(24) = 60 + 168$}
    \item \textcolor{green}{$s_a(24) = 228$}
  \end{enumerate}

  \column{0.5\textwidth}
  \begin{enumerate}[<+(1)->]
    \item \textcolor{green}{$s_b(x) = 70 + 8x$}
    \item \textcolor{green}{$s_b(24) = 70 + 8\cdot 24$}
    \item \textcolor{green}{$s_b(24) = 70 + 192$}
    \item \textcolor{green}{$s_b(24) = 262$}
    \item \textcolor{green}{Team B wins!}
  \end{enumerate}
  \end{columns}
\end{frame}

\end{document}


