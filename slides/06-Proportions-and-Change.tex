\documentclass[]{beamer}
\mode<presentation>
{
  \usetheme{Warsaw}
  \definecolor{mcgarnet}{rgb}{0.38, 0, 0.08}
  \definecolor{mcgray}{rgb}{0.6, 0.6, 0.6}
  \setbeamercolor{structure}{fg=mcgarnet,bg=mcgray}
  %\setbeamercovered{transparent}
}


\usepackage[english]{babel}
\usepackage[latin1]{inputenc}
\usepackage{times}
\usepackage[T1]{fontenc}
\usepackage{tikz}
\usepackage{graphicx}

\newcommand{\imagesource}[1]{{\centering\hfill\break\hbox{\scriptsize Image Source:\thinspace{\small\itshape #1}}\par}}

\title{06 - Proportions and Change}


\author{Dr. Robert Lowe\\}

\institute[Maryville College] % (optional, but mostly needed)
{
  Division of Mathematics and Computer Science\\
  Maryville College
}

\date[]{}
\subject{}

\pgfdeclareimage[height=0.5cm]{university-logo}{images/Maryville-College}
\logo{\pgfuseimage{university-logo}}



\AtBeginSection[]
{
  \begin{frame}<beamer>{Outline}
    \tableofcontents[currentsection]
  \end{frame}
}


\begin{document}

\begin{frame}
  \titlepage
\end{frame}

\begin{frame}{Outline}
  \tableofcontents
\end{frame}


% Structuring a talk is a difficult task and the following structure
% may not be suitable. Here are some rules that apply for this
% solution: 

% - Exactly two or three sections (other than the summary).
% - At *most* three subsections per section.
% - Talk about 30s to 2min per frame. So there should be between about
%   15 and 30 frames, all told.

% - A conference audience is likely to know very little of what you
%   are going to talk about. So *simplify*!
% - In a 20min talk, getting the main ideas across is hard
%   enough. Leave out details, even if it means being less precise than
%   you think necessary.
% - If you omit details that are vital to the proof/implementation,
%   just say so once. Everybody will be happy with that.

\section{Change According to Proportions and Percents}
\begin{frame}{Direct Proportions}
\begin{itemize}[<+->]
    \item When something increases by a proportion, this is called a direct proportion.
    \item Suppose we have a proportion $a:b :: c:d$.  If an increase in $a$ causes a proportional increase in $c$, then $a:b :: c:d$ is a direct proportion.  
    \item You must read the nature of a problem to know whether it is increasing and therefore a direct proportion.
\end{itemize}
\end{frame}

\begin{frame}{Inverse Proportions}
\begin{itemize}[<+->]
    \item When something decreases by a proportion, this is called an inverse proportion.
    \item Using the letters $a$, $b$, $c$, and $d$ from the previous problem, if an increase in $a$ causes a decrease in $c$, then the corresponding inverse proportion is $a:b :: d:c$.
    \item You must read the nature of a problem to know whether it is decreasing and therefore an inverse proportion.
\end{itemize}
\end{frame}

\begin{frame}{Increasing, Amount, and Markup}

    Recall the term amount.  
    
    \[
        \mathrm{amount} = \mathrm{base} + \mathrm{percentage}
    \]
\end{frame}

\begin{frame}{Decreasing, Difference, Discount}

Recall the term difference.  
\[
    \mathrm{difference} = \mathrm{base} - \mathrm{percentage}
\]
\end{frame}


\section{Percentage and Proportion Problems}

\begin{frame}{Problem 3}

  If a staff of $4\mathrm{ft}$ casts a shadow $7\mathrm{ft}$ in length, what is the height of a tower which casts a shadow of $198\mathrm{ft}$ at the same time? 

\end{frame}

\begin{frame}{Problem 4}

  A homeowner sells their house at a loss of $20\%$.  If the selling price was 
  $\$60,000.00$, what was the original price of the home?

\end{frame}

\begin{frame}{Problem 5}

  In the erection of a house I paid twice as much for material
  as for labor.  Had I paid $6\%$ more for material, and $9\%$ more
  for labor, my house would have cost $\$1284.00$; what was its
  cost?
\end{frame}


\end{document}


