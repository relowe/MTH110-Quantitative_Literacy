\documentclass[]{beamer}
\mode<presentation>
{
  \usetheme{Warsaw}
  \definecolor{mcgarnet}{rgb}{0.38, 0, 0.08}
  \definecolor{mcgray}{rgb}{0.6, 0.6, 0.6}
  \setbeamercolor{structure}{fg=mcgarnet,bg=mcgray}
  %\setbeamercovered{transparent}
}


\usepackage[english]{babel}
\usepackage[latin1]{inputenc}
\usepackage{times}
\usepackage[T1]{fontenc}
\usepackage{tikz}
\usepackage{graphicx}
\usepackage{xcolor}
\usepackage{amsmath}

\newcommand{\imagesource}[1]{{\centering\hfill\break\hbox{\scriptsize Image Source:\thinspace{\small\itshape #1}}\par}}

\title{04 - Ratio and Proportion}


\author{Dr. Robert Lowe\\}

\institute[Maryville College] % (optional, but mostly needed)
{
  Division of Mathematics and Computer Science\\
  Maryville College
}

\date[]{}
\subject{}

\pgfdeclareimage[height=0.5cm]{university-logo}{images/Maryville-College}
\logo{\pgfuseimage{university-logo}}



\AtBeginSection[]
{
  \begin{frame}<beamer>{Outline}
    \tableofcontents[currentsection]
  \end{frame}
}


\begin{document}

\begin{frame}
  \titlepage
\end{frame}

\begin{frame}{Outline}
  \tableofcontents
\end{frame}


% Structuring a talk is a difficult task and the following structure
% may not be suitable. Here are some rules that apply for this
% solution: 

% - Exactly two or three sections (other than the summary).
% - At *most* three subsections per section.
% - Talk about 30s to 2min per frame. So there should be between about
%   15 and 30 frames, all told.

% - A conference audience is likely to know very little of what you
%   are going to talk about. So *simplify*!
% - In a 20min talk, getting the main ideas across is hard
%   enough. Leave out details, even if it means being less precise than
%   you think necessary.
% - If you omit details that are vital to the proof/implementation,
%   just say so once. Everybody will be happy with that.

\section{Ratios}
\begin{frame}{A Complex Sounding Problem}
If 8 workers in 24 days working 10 hours a day can reap 48 acres of
wheat, how many acres could 12 workers reap in 20 days of 12 hours
each?
\end{frame}

\begin{frame}{Comparing and Relating Quantities}
\begin{itemize}[<+->]
    \item A ratio compares quantities of like types. (days to days, dollars to dollars, 
    workers to workers, etc.)
    \item Ratios express the relationship between two concrete quantities.
\end{itemize}
\end{frame}

\begin{frame}{Notation}
\begin{itemize}[<+->]
    \item The notation for a ratio is $a:b$ Where $a$ is the first
        quantity and $b$ is the second.
    \item This is frequently pronounced as ``$a$ to $b$''
    \item A ratio is analgous to a fraction, thus $1:2$, $1\div 2$,
        $\frac{1}{2}$, and $0.5$ are all the same ratio.
\end{itemize}
\end{frame}

\begin{frame}{Ratios, Categories, and Properties}
\begin{itemize}
    \item<2-> Often we use ratios to categorize populations of similar items.
    \item<3-> Example:  What is the ratio of women to men in this room? What is the ratio of men to women?
        \newline\uncover<4->{\color{green}$9 \mathrm{men} : 17 \mathrm{women}$}
    \item<5-> Ratios can be reduced in the same way we reduce fractions.
    \item<6-> Ratios can always be compared, even when they represent ratios of disparate objects.
    \item<7-> Ratios are abstract numbers. Why?
\end{itemize}
\end{frame}


\section{Proportion}
\begin{frame}{Proportion}
\begin{itemize}[<+->]
  \item A proportion is two ratios which represent the same fraction.  (Example: $1:2$ and $2:4$.)
  \item We often use the word ``in proportion'' to describe two equal ratios.
  \item Proportions are the mathematical equivalent of analogies: $a$ is to $b$ as $c$ is to $d$.
\end{itemize}
\end{frame}


\begin{frame}{Proportion Notation and Properties}
\begin{itemize}[<+->]
    \item There are two main ways to write proportions $a:b :: c:d$ or $a:b = c:d$ where $a$, $b$, $c$, and $d$ are the numbers which make up the proportion.  Example: $1:2::2:4$ or $1:2=2:4$.
    \item The two outer numbers are called the {\bf extremes} of the proportion.  $a:b::c:d$ has extremes $a$ and $d$.
    \item The two inner numbers are called the {\bf means} of the proportion.  $a:b::c:d$ has means $b$ and $c$.
    \item In order for four numbers to be in proportion, the product of the extremes must equal the product of the means.  So in $a:b::c:d$, $a\times d = b \times c$.
    \item Examples: $1:2::2:4$, $1:3::6:18$
    \item Discuss: Why must these two products be the same?
\end{itemize}
\end{frame}

\begin{frame}{The Rule of Three}
\begin{itemize}
    \item If three numbers of a proportion are known, the fourth may be found.  
    \item Missing Mean (if we know $a$, $c$, and $d$)
    \begin{align*}
        \uncover<2->{a:x &:: c:d\\}
        \uncover<3->{x\cdot c &= a \cdot d\\}
        \uncover<4->{x &=\dfrac{a\cdot d}{c}}
    \end{align*}
    
    \item Missing Extreme (if we know $b$, $c$, $d$)
    \begin{align*}
        \uncover<5->{x:b &:: c:d\\}
        \uncover<6->{x\cdot d &= b \cdot c \\}
        \uncover<7->{x &= \dfrac{b \cdot c}{d}}
    \end{align*}
\end{itemize}
\end{frame}

\begin{frame}{Example 1}
    Assuming that all classes maintain the same proportion of
    men and women as this one, if a class had 20 men, how many women
    would it have?

    {\color{green}
    \begin{align*}
        \uncover<2->{9:17 &:: 20:x\\}
        \uncover<3->{17 \cdot 20 &= 9x\\}
        \uncover<4->{340 &=9x\\}
        \uncover<5->{\dfrac{340}{9} &= \dfrac{9x}{9}\\}
        \uncover<6->{38 &\approx x\\}
    \end{align*}}

\end{frame}

\begin{frame}{Example 2}
    A besieged town, containing 22,400 inhabitants, has provisions to
    last 3 weeks; how many must be sent away that they may be able to
    hold out 7 weeks? \footnotesize{Transcribed from: {\em A Treatise
    on Arithmetic} by J. H. Smith. 1878}
    {\color{green}
    \begin{align*}
        \uncover<2->{22,400 : x &:: 7:3\\}
        \uncover<3->{7x &= 3(22,400)\\}
        \uncover<4->{7x &= 67,200\\}
        \uncover<5->{\dfrac{7x}{7} &= \dfrac{67,200}{7}\\}
        \uncover<6->{x &= 9,600\\}
        \uncover<7->{22,400 - 9,600 &= 12,800\\}
    \end{align*}
        \uncover<8->{12,800 people must be sent away.}
    }
\end{frame}

\begin{frame}{Compound Proportions}
\begin{itemize}
    \item A compound proportion is a set of three or more ratios given where one is incomplete.
    \item You produce solutions to compound proportions by multiplying corresponding terms together, and then solve 
    as in a simple proportion.
    \item Our opening example is a compound proportion.
\end{itemize}
\end{frame}

\begin{frame}{A Complex Sounding Problem - Solved}
If 8 workers in 24 days working 10 hours a day can reap 48 acres of
wheat, how many acres could 12 workers reap in 20 days of 12 hours
each?

{\color{green}
\[
    \left.
    \begin{array}{rrcr}
        \uncover<3->{\mathrm{workers} & 8 & : & 12\\}
        \uncover<4->{\mathrm{days} & 24 & : & 20\\}
        \uncover<5->{\mathrm{hours} & 10 & : & 12\\}
    \end{array}
    \right\}
    \uncover<2->{:: 48 : x}
\]
}
\end{frame}

\end{document}


