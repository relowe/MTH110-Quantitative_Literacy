% Don't touch this %%%%%%%%%%%%%%%%%%%%%%%%%%%%%%%%%%%%%%%%%%%
\documentclass[11pt]{article}
\usepackage{fullpage}
\usepackage[left=1in,top=1in,right=1in,bottom=1in,headheight=3ex,headsep=3ex]{geometry}
\usepackage{graphicx}
\usepackage{float}
\usepackage{longtable}

\newcommand{\blankline}{\quad\pagebreak[2]}
%%%%%%%%%%%%%%%%%%%%%%%%%%%%%%%%%%%%%%%%%%%%%%%%%%%%%%%%%%%%%%

% Modify Course title, instructor name, semester here %%%%%%%%


\title{MTH 110: Quantitative Literacy}
\author{Dr. Robert Lowe}
\date{Spring, 2020}

%%%%%%%%%%%%%%%%%%%%%%%%%%%%%%%%%%%%%%%%%%%%%%%%%%%%%%%%%%%%%%

% Don't touch this %%%%%%%%%%%%%%%%%%%%%%%%%%%%%%%%%%%%%%%%%%%
\usepackage[sc]{mathpazo}
\linespread{1.05} % Palatino needs more leading (space between lines)
\usepackage[T1]{fontenc}
\usepackage[mmddyyyy]{datetime}% http://ctan.org/pkg/datetime
\usepackage{advdate}% http://ctan.org/pkg/advdate
\newdateformat{syldate}{\twodigit{\THEMONTH}/\twodigit{\THEDAY}}
\newsavebox{\MONDAY}\savebox{\MONDAY}{Mon}% Mon
\newcommand{\week}[1]{%
%  \cleardate{mydate}% Clear date
% \newdate{mydate}{\the\day}{\the\month}{\the\year}% Store date
  \paragraph*{\kern-2ex\quad #1, \syldate{\today} - \AdvanceDate[4]\syldate{\today}:}% Set heading  \quad #1
%  \setbox1=\hbox{\shortdayofweekname{\getdateday{mydate}}{\getdatemonth{mydate}}{\getdateyear{mydate}}}%
  \ifdim\wd1=\wd\MONDAY
    \AdvanceDate[7]
  \else
    \AdvanceDate[7]
  \fi%
}
\usepackage{setspace}
\usepackage{multicol}
%\usepackage{indentfirst}
\usepackage{fancyhdr,lastpage}
\usepackage{url}
\pagestyle{fancy}
\usepackage{hyperref}
\usepackage{lastpage}
\usepackage{amsmath}
\usepackage{layout}

\lhead{}
\chead{}
%%%%%%%%%%%%%%%%%%%%%%%%%%%%%%%%%%%%%%%%%%%%%%%%%%%%%%%%%%%%%%

% Modify header here %%%%%%%%%%%%%%%%%%%%%%%%%%%%%%%%%%%%%%%%%
\rhead{\footnotesize MTH1100-01 Spring 2020}

%%%%%%%%%%%%%%%%%%%%%%%%%%%%%%%%%%%%%%%%%%%%%%%%%%%%%%%%%%%%%%
% Don't touch this %%%%%%%%%%%%%%%%%%%%%%%%%%%%%%%%%%%%%%%%%%%
\lfoot{}
\cfoot{\small \thepage/\pageref*{LastPage}}
\rfoot{}

\usepackage{array, xcolor}
\usepackage{color,hyperref}
\definecolor{clemsonorange}{HTML}{EA6A20}
\hypersetup{colorlinks,breaklinks,linkcolor=clemsonorange,urlcolor=clemsonorange,anchorcolor=clemsonorange,citecolor=black}

\begin{document}

\maketitle

\blankline

\begin{tabular*}{.93\textwidth}{@{\extracolsep{\fill}}lr}

%%%%%%%%%%%%%%%%%%%%%%%%%%%%%%%%%%%%%%%%%%%%%%%%%%%%%%%%%%%%%%

% Modify information %%%%%%%%%%%%%%%%%%%%%%%%%%%%%%%%%%%%%%%%%
E-mail: \texttt{robert.lowe@maryvillecollege.edu} & Office Phone: 865-981-8169 \\

 Office Hours: MWF 1:00PM -- 2:00PM, TR 3:00PM -- 4:00PM  &  Class Hours: MWF 9:00 -- 9:50\\

 Office: SSC 214 & Class Room: SSC 231\\
\hline
\end{tabular*}

\vspace{5 mm}

% First Section %%%%%%%%%%%%%%%%%%%%%%%%%%%%%%%%%%%%%%%%%%%%

\section*{Course Description}

In a world of accelerating change, productive citizenship and leadership continue to require greater
fluency with quantitative information and reasoning. The goals of courses in the quantitative literacy
domain are to promote a way of understanding the world that uses data and the quantitative analysis of
data to make interpretations, arguments, and conclusions.

% Second Section %%%%%%%%%%%%%%%%%%%%%%%%%%%%%%%%%%%%%%%%%%%

\section*{Required Materials}

\begin{itemize}
\item A Scientific Calculator (Graphing Calculators are fine, but no cell phones/tablets may be used during exams or quizzes!)
\item Various internet resources and handouts
\item A laptop/chromebook is recommended.  All of the software which we will use is accessible through a web browser.  Tablet computers can also be used with varying degrees of success. 

\end{itemize}
NOTE: {\em Math and You} by Ron Larson is specified as a text for this course.  We will
only be using this in a limited way (if at all).  You can access this text online, for free. You do not need to purchase a copy of this book.

% Third Section %%%%%%%%%%%%%%%%%%%%%%%%%%%%%%%%%%%%%%%%%%%

\section*{Prerequisites/Corequisites}
Math 105 or satisfactory performance on the first year/transfer
Mathematics Assessment.

% Fourth Section %%%%%%%%%%%%%%%%%%%%%%%%%%%%%%%%%%%%%%%%%%%
\section*{Core Curriculum Information}

This course meets the Quantitative Literacy domain requirements in the Core Curriculum; passing this course will fulfill a student's general education requirement in that domain. Accordingly, it addresses a major educational goal of the Core Curriculum, ``proficiency in the use of evidence, empirical data, and quantitative analysis'', by seeking to develop quantitative literacy.

By completing a Quantitative Literacy course, students will demonstrate the ability to:
\begin{enumerate}
    \item Interpret quantitative information, arguments, and claims.

    \item Formulate logical arguments supported by quantitative evidence.

    \item Communicate quantitative information using various mathematical forms (e.g. words, tables, graphs, diagrams, and equations).

    \item Solve quantitative problems that arise within the contexts of civic, professional, and personal life.

    \item Draw appropriate conclusions based on quantitative analysis.
\end{enumerate}

% Fifth Section %%%%%%%%%%%%%%%%%%%%%%%%%%%%%%%%%%%%%%%%%%%
\section*{Course Structure}

\subsection*{Methods of Instruction}
Classroom instruction will include lecture; discussion; individual,
pair, and group problem solving; and peer presentation. Active participation is expected in every class period. In addition to classroom activities, every week will bring a new set of problems and assigned readings.  Full participation in homework and reading is expected.

\subsection*{Technology}
Technology will be utilized frequently in this class. It is recommended that you bring a calculator to
class each day. It will also be helpful for all students with laptops to bring them to class; Excel and/or Google Sheets will be
used as needed.

% Course Schedule %%%%%%%%%%%%%%%%%%%%%%%%%%%%%%%%%%%%%%%%%%%

\newpage
\subsection*{Schedule}

This is the tentative schedule for our course.  There may be some
slight modifications to the following according to the needs of the
semester. However, the exam dates are fixed, and will be followed.
All exam dates are shown in bold font, and it is absolutely vital to
your success in this course that you attend the exam days.  The final
exam period is set by the registrar.  Attendance on the date of the
final exam is absolutely mandatory; any student failing to appear on
this date will receive a failing grade for the course.

\subsubsection*{January 2020}
\begin{tabular}{rrrrrrr}
Su & Mo & Tu & We & Th & Fr & Sa\\
   &    &    &  1 &  2 &  3 &  4\\ 
 5 &  6 &  7 &  8 &  9 & 10 & 11\\ 
12 & 13 & 14 & 15 & 16 & 17 & 18\\ 
19 & 20 & 21 & 22 & 23 & 24 & 25\\ \cline{6-6}
26 & 27 & 28 & 29 & 30 & \multicolumn{1}{|r|}{31} &\\ \cline{6-6}
\end{tabular}

\begin{itemize}
\item\textbf{Wed January  8}
  - Notation, Numbers, \& Operations - {\em Pretest}
\item\textbf{Fri January 10}
  - Notation with Large and Small Quantities
\item\textbf{Mon January 13}
  - Significance and rounding - {\em Homework 1 Due}
\item\textbf{Wed January 15}
  - Formulae and Functions - {\em Quiz 1}
\item\textbf{Fri January 17}
  - Proportions
\item\textbf{Mon January 20} 
  - Standardized Proportions \& The Number 100 - {\em Homework 2 Due}
\item\textbf{Wed January 22}
  - Proportions of Change - {\em Quiz 2}
\item\textbf{Fri January 24}
  - Proportions and Prices  
\item\textbf{Mon January 27}
  - Unit Conversions - {\em Homework 3 Due}
\item\textbf{Wed January 29}
  - Review - {\em Quiz 3}
\item\textbf{Fri January 31}
  - Exam 1
\end{itemize}
\hrulefill

\subsubsection*{February 2020}
\begin{tabular}{rrrrrrr}
Su & Mo & Tu & We & Th & Fr & Sa\\
   &    &    &    &    &    &  1\\ 
 2 &  3 &  4 &  5 &  6 &  7 &  8\\ 
 9 & 10 & 11 & 12 & 13 & 14 & 15\\ 
16 & 17 & 18 & 19 & 20 & 21 & 22\\ \cline{6-6}
23 & 24 & 25 & 26 & 27 & \multicolumn{1}{|r|}{28} & 29\\ \cline{6-6}
\end{tabular}
\begin{itemize}
\item\textbf{Mon February  3}
  - Inflation and You - Computing CPI
\item\textbf{Wed February  5}
  - CPI and Price Conversion - {\em Quiz 4}
\item\textbf{Fri February  7}
  - Financial Mathematics 
\item\textbf{Mon February 10}
  - Introduction to Spreadsheets \& Budgeting - {\em Homework 4}
\item\textbf{Wed February 12}
  - Spreadsheet Tutorials - {\em In Class Spreadsheet Assignment} 
\item\textbf{Fri February 14}
  - Spreadsheets \& Graphing  
\item\textbf{Mon February 17}
  - Linear Functions: Identifying and Extraction - {\em Homework 5 Due}
\item\textbf{Wed February 19}
  - Graphing and Projection Using Linear Functions - {\em Quiz 5} 
\item\textbf{Fri February 21}
  - Taxes: Income, Sales, and Excise 
\item\textbf{Mon February 24}
  - Debt \& Simple Interest  - {\em Homework 6 Due}
\item\textbf{Wed February 26}
  - Review - {\em Quiz 6}
\item\textbf{Fri February 28}
  - Exam 2 
\end{itemize}
\hrulefill

\subsubsection*{March 2020}
\begin{tabular}{rrrrrrr}
Su & Mo & Tu & We & Th & Fr & Sa\\
 1 &  2 &  3 &  4 &  5 &  6 &  7\\ 
 8 &  9 & 10 & 11 & 12 & 13 & 14\\ 
15 & 16 & 17 & 18 & 19 & 20 & 21\\ 
22 & 23 & 24 & 25 & 26 & 27 & 28\\
29 & 30 & 31 &    &    &    & \\
\end{tabular}

\begin{itemize}
\item\textbf{Mon March  2}
  - Exponential Functions
\item\textbf{Wed March  4}
  - Project 2 Presentations - {\em Project 2 Due}
\item\textbf{Fri March  6}
  - Project 2 Presentations 
\item\textbf{Mon March  9}
  - Exponential Growth and Decay - {\em Homework 7 Due}
\item\textbf{Wed March 11}
  - Compound Interest  - {\em Quiz 7}
\item\textbf{Fri March 13}
  - Compound Interest and Annuities
\item\textbf{Mon March 16} Spring Break
\item\textbf{Wed March 18} Spring Break
\item\textbf{Fri March 20} Spring Break
\item\textbf{Mon March 23}
  - Amortization of Loans  - {\em Homework 8 Due}
\item\textbf{Wed March 25}
  - Measuring Likelihood - {\em Quiz 8}
\item\textbf{Fri March 27}
  - Disjoint Probabilities
\item\textbf{Mon March 30}
  - Independent Probabilities
\end{itemize}
\hrulefill


\subsubsection*{April 2020}
\begin{tabular}{rrrrrrr}
Su & Mo & Tu & We & Th & Fr & Sa\\
   &    &    &  1 &  2 &  3 &  4\\
 5 &  6 &  7 &  8 &  9 & 10 & 11\\ \cline{6-6}
12 & 13 & 14 & 15 & 16 & \multicolumn{1}{|r|}{17} & 18\\ \cline{6-6}
19 & 20 & 21 & 22 & 23 & \multicolumn{1}{|r|}{24} & 25\\ \cline{6-6}
26 & 27 & 28 & 29 & 30 &    &\\ 
\end{tabular}
\begin{itemize}
\item\textbf{Wed April  1}
  - The Monte Hall Problem \& Other Puzzles - {\em Quiz 9}
\item\textbf{Fri April  3}
  - Describing Data: Mean, Median, and Mode 
\item\textbf{Mon April  6}
  - Distributions of Data: Five Number Summary - {\em Homework 9 Due}
\item\textbf{Wed April  8}
  - Measuring Variance About the Mean - {\em Quiz 10}
\item\textbf{Fri April 10} Good Friday - College Closed
\item\textbf{Mon April 13}
  - Histograms and Skew  - {\em Homework 10 Due}
\item\textbf{Wed April 15}
  - 68, 95, 99.7 Rule 
\item\textbf{Fri April 17}
  - Exam 3
\item\textbf{Mon April 20}
  - Project 3 Debate - {\em Project 3 Due}
\item\textbf{Wed April 22}
  - Review for Final Exam
\item\textbf{Fri April 24 9:00AM} - {\em Final Exam}
\end{itemize}
\hrulefill


\subsection*{Grade Calculation}
This course is graded using a weighted average among six categories.  The assignments 
within each category are equally weighted and are all graded out of 100 points.  Hence your final numeric grade is computed by
finding the average of each category, and then multiplying them by the corresponding weight. The
weights for each category are as follows:

\begin{itemize}
    \item Attendance 15\%
    \item Homework 5\%
    \item Quizzes 15\%
    \item Projects 30\%
    \item Regular Exams 15\%
    \item Comprehensive Final 20\%
\end{itemize}

As you complete each assignment, a report of each grade will be given to you.  We will
not, however, use the Tartan to track your average.  Instead, tracking your averages
is your responsibility.  In fact, you will be periodically asked to compute your own grade
as part of your homework assignments.  In this way, you will be graded on how well you keep track of your
own grade.

Letter grades will be assigned according to the following scale:

\begin{tabular}{|lr|lr|lr|lr|lr|}
    \hline
    A+ & 96.7--100\% & B+ & 86.7--90\% & C+ & 76.7--80\% & D+ & 66.7--70\% & F & less than 60 \% \\
    A  & 93.3--96.7\% & B & 83.3--86.7\% & C & 73.3--76.7\% & D & 63.3--66.7\% & & \\
    A- & 90--93.3\% & A- & 80--83.3\% & C- & 70--73.3\% & D- & 60--63.3\% & & \\
    \hline
\end{tabular}


\subsection*{Assessments}
The standards of assessment in each grading category will be as follows. 

\subsubsection*{Attendance {\em (15\% of the final grade)}}
Attendance in this class is
mandatory, and attendance is defined as full participation for the
entire duration of a class period. Attendance will be taken at the
end of every class period. Partial attendance for a class meeting
will receive no credit. All homework assignments and project write-ups
will be due at the beginning of their respective class period and all
quizzes will be given during the first few minutes of their respective
class periods. Failure to submit an assignment and/or failure to take
a quiz will also result in an absence for the day.

All assignments will be given to students in hard-copy format during
class. No electronic documents will be posted, nor will any be
emailed to students. If you fail to attend a class meeting, it is
your responsibility to obtain a copy of the assignment from either
a classmate or from your professor. Absence on the day an assignment
is given will neither excuse you from the assignment nor will it
result in an extension.

\subsubsection*{Homework {\em(5\% of the final grade)}}
Homework will be assigned each
week, and will be due at the beginning of the first class period of
the week. Homework must be neatly handwritten and your name must
appear on the first page of your assignment. If your homework
comprises multiple pages, you must staple them together prior to
entering the classroom. Any assignment which does not bear
a student's name will be shredded, and you will receive a zero for
that assignment (as well as an absence for that day.) Any loose pages,
other than the first page of an assignment will also be destroyed.

Homework will be graded as follows. 80\% of the homework grade will be
based on completion and 20\% will be based on a selected problem's
correctness. Completed problems consist of the following: A copy of
the problem statement, handwritten, all steps required to set up and
solve the problem, and then the final answer. Problems which are answered
with simply a number will receive no credit. A homework assignment
which receives a grade of zero will result in an unexcused class absence for
the day on which it is due.

\subsubsection*{Quizzes {\em(15\% of the final grade)}}
Weekly quizzes will be given at
the beginning of the second class period of the week.  Failure to be present
during a quiz will result in an absence for the day and a zero for
the quiz. No makeup quizzes will be given except as noted below in
the excused absences section. Quizzes will cover the previous week's
material and details found in the assigned reading.

\subsubsection*{Projects {\em (30\% of the final grade)}}
There will be three projects
assigned during the semester. All three projects will require both
application of techniques covered in class and independent research.
The first and third projects will result in an essay of no fewer than 1500 words and no more than 5000 words. We will use MLA citations as our citation standard.  Your essays will describe your techniques and report your
findings. This essay will be graded on content, grammar, and
reliability of cited sources. Great care must be exercised to not
plagiarize sources, and plagiarism will not be tolerated in any form.

Project essays will be type written, and will be due in hard copy form
at the beginning of a class period. Failing to submit your essay will
result in an absence for that day and a grade of zero for the project. No electronic submissions will be
accepted, and all handwritten submissions will be shredded.

The second project will include a class presentation.

\subsubsection*{Regular Exams {\em (15\% of the final grade)}}
Three exams will be given
throughout the semester. Each will cover material up to that point in
the course. Exam dates are listed in the course schedule, and failure to
attend an exam period (excepting as noted in the excused absences
policy), will result in an irrevocable zero for the exam.

\subsubsection*{Comprehensive Final {\em (20\% of the final grade)}} 
The comprehensive
final exam will be given on the day and time specified by the
registrar (which is listed in the course schedule in this syllabus). This exam will cover all of the material for the semester.
Except in extreme circumstances, the final exam date shall be
immovable. The final exam is mandatory, failure to appear on the day of the final
exam will result in a failing grade for the course.

% Fifth Section %%%%%%%%%%%%%%%%%%%%%%%%%%%%%%%%%%%%%%%%%%%

\newpage
\section*{Course Policies}

\subsection*{Late Policy}
No late work will be accepted under any circumstances (except as mercy
and decency may dictate in extremely rare events).

\subsection*{Extra Credit}
No extra credit will be given under any circumstances.

\subsection*{Excused Absences}
In some cases, absences may be excused. These include:
\begin{itemize}
    \item School Sanctioned Events (Sports, Concerts, etc.)
    \item Severe Illness
    \item Family Emergencies
    \item Court Appearance / Jury Duty
\end{itemize}
In the case of a school event, notice must be given at least one week
prior to the absence. The notice must include a signed note from the
faculty or staff member in charge of the event. This note must be
given in physical form, electronic notes will not be accepted.
In the case of illness, a doctor's note is required. Note that
except in extreme circumstances, doctor's appointments do not qualify as a valid reason to miss
a class. Please be respectful of the other students, and schedule
appointments during your free time.

Family emergencies will require some form of proof. Where possible,
you must give advance notice of missing a class. The exception to this
would need to be fairly severe, and hopefully it will not come up.
For court appearances and/or jury duty, you must provide a copy of
your summons. You may redact any details you wish, save for the
actual date and time of your appearance. Court appearances must be
cleared at least one week in advance.

\subsection*{Making Up Excused Absences}
Should you be in a situation in which you receive an excused absence,
this in no way will extend your due dates (excepting extreme
emergencies). You must make up any quiz or test at a designated time 
prior to your excused absence. Also, homework or projects must
be submitted prior to the class period in which they are due.

\subsection*{Communication and Extra Help}
You are always welcome at office hours for help with any
questions you may have about the course. For help at other times during the day, stop by or call my office to see if I'm available. You can also contact me by email, but often I can better help you face to face and may respond with a request that
you come to see me. Note that I do not typically respond to email between 5 p.m. and 8 a.m. You may make appointments to see me at other times if your schedule does not permit you to attend my office hours.

Another way to get help is via our GroupMe chat group.  Feel free to use this to converse with each other and/or ask me for help in the group.  If you have a question about a problem, chances are strong that another student has that same question.  If you use the group chat, you will help everyone. (Though please keep in mind that all communications in the group are visible to all members of the group.) 

\subsection*{Plagiarism and Cheating}
You are expected to do your own work. Never submit work of others,
never give unauthorized assistance to others, do not use unauthorized aids during exams and do not ask for help from other
faculty members without the approval of your professor. Plagiarism and cheating are serious offenses that will not be
tolerated. Explanations regarding these offenses and how they are handled can be found in the MC Student Handbook at\newline
https://www.maryvillecollege.edu/academics/catalog/handbook/section-nine/.\newline
You are expected to have read and understand these policies. Offenses on specific assignments, quizzes, or exams will result
in a score of 0 on the relevant assignment, and a letter of censure will be placed in your college file. Repeat offenses will
result in further disciplinary action, including the possibility of failing the course.

\subsection*{Students with Disabilities}
Any student who feels s/he may need learning or physical
accommodation(s) based on the impact of a disability should contact Services for Students with Disabilities to discuss your
specific needs. Please contact 981-8124 to coordinate reasonable accommodations for students with documented
disabilities. The Disability Services office is located in the Learning Center in the basement of Thaw Hall. Undocumented
disabilities will not be accommodated.

\end{document}
