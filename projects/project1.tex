\documentclass{article}
\usepackage{fancyhdr}
\usepackage{epigraph}

\pagestyle{fancy}
\fancyhf{}
\lhead{Math 110}
\chead{Project 1}
\rhead{Due: February 8, 2019}
\cfoot{\thepage}

\begin{document}
    \epigraph{Destiny is not a matter of chance, it is a matter of choice.}{\textit{William Jennings Bryan}}

\section*{Introduction}
The choices you make now (to be at MC, to attend class, to study, etc.) obviously have great bearing 
on your future! As you consider the worthiness of your endeavors, you might also consider the 
financial cost of your choice to be here at MC. The first part of this activity will lead you to investigate 
the cost of each session of class you attend while the remaining parts will help you to evaluate the 
long term ramifications of your decisions. NOTE: 
All calculations will be based on your expected 
spring schedule.

\section*{Submission Requirements}
This project requires you to write an essay.  Your essay must be between 1,500 and 5,000 words in length.  The formatting requirements for this essay are as follows:
\begin{itemize}
    \item Typewritten, double spaced, and 12 point font.
    \item The first line of the first page consists of your title centered on a line by itself.
    \item The second line of the first page consists of your name centered on a line by itself.
    \item Each page following the first page has a header on the right hand side consisting of your last name and the page number.  (No header appears on the first page.)
    \item All cited sources will be cited using MLA format.\newline
    (See {\tt http://www.bibme.org/mla} for more information for the citing standard.)
    \item The last page of your essay will contain the
    works cited section.  This section should be indicated with a boldfaced heading centered on a line by itself.
    \item Because this essay is a personal narrative, you may write sections in the first person where appropriate.
\end{itemize}

You will be graded on adherence to formatting standards, grammar, content, and the mathematical soundness of your work.  As you proceed through the sections of this project, be sure to keep notes about your calculations and procedures as these should be detailed in your essay.  Only a physical printout will be accepted, and it must be submitted on the designated due date in order to receive credit.

\section*{Project Overview}
Your essay will address the following questions:
\begin{itemize}
\item What is the total cost of earning your degree?
\item What factors influence the cost of your degree?
\item How much does each credit hour cost?
\item How much does each class session cost?
\item Is the expense worthwhile?  That is, does the expected outcome of your education increase your expected life outcome sufficiently to compensate you for the expense of obtaining your degree?
\end{itemize}

The manner in which you answer these questions are up to you, however advice will be given in the following sections.  Your essay should take a narrative
approach, detailing how you obtained the answers to each question, the mathematics involved, and should have at least 4 cited sources.  One source is suggested in each section of this writeup, the rest will be up to you to locate.

\section*{Analyzing the Cost of College}
The first consideration for this project is the cost of your college education.  The suggested way to analyze your cost is by picking a semester and then using that as a ``typical'' semester.  For our purposes, use the current semester as your basis.  We will also make the assumption that the cost of each semester will remain constant.  You should be able to access this information via self service.  

As you would have to cover living expenses (food, housing, etc.) even if you were not a student, we will exclude those from our analysis.  Focus in on the expenses associated directly with the activity of being a student at Maryville college.  These expenses will include items such as tuition, the cost of your textbooks, fees, course specific fees, parking permits, and school supplies.  Try to make as accurate an accounting as you can.  You will also want to note the number of credit hours you are taking this semester.  As Maryville college requires 120 credit hours to graduate, how many semesters like the present one will be needed to graduate?  Note that as you do this, we are not looking at how you are covering your expenses.  Scholarships, grants, and other forms of financial aid do not affect the cost of your education!

Now, compute your overall cost by taking your present expenses and multiplying by the expected length of your sojourn in your alma mater.  This will give you your overall expected cost for your degree.

For working out the cost of a credit hour, you need only consider the total number of credits you are taking.  You can do this either using your total expenses and the 120 credit hours which comprise your undergraduate career, or you can use the total expenses for this semester and the number of credit hours you are presently attempting.

The final calculation to be made in this section is the cost of each class meeting.  That is to say, every time you enter a classroom (or fail to do so when you should), how much money have you spent?  To accomplish this, you will want to work out the total number of meetings for all of your classes.  This will make a TR class seem more expensive per meeting than a MWF class.  Is this fair?  Be sure to discuss why or why not.  In the end, you will have a dollar amount for what each meeting costs.  Our hope in having you do this is that you will feel the pain of wasted money whenever you elect to engage in some less important activity during class time.

As you write this section of your essay, be sure to include an accounting of the information you gathered.  Be sure to include a discussion about how these expenses may vary from student to student.  To do this, consider your own expenses, and how those may be different for others.  You also need to give an account of the techniques you used to solve the problem, including a nice typesetting of your calculations.  You may also want to discuss anything in these results that you found surprising.  Another exploration is to relate the money you waste when you skip class to expensive goods you may want to purchase.  For example, how many missed class meetings does it take before you have thrown away enough money to purchase a Nintendo\textsuperscript{tm} Switch?

\section*{Analyzing the Benefits of College}
Now that we know your education is tremendously expensive, we will now explore
whether or not you will realize an equal or greater benefit for having spent
so much time and money acquiring your education.

Start by reading the following article:\newline
{\footnotesize\tt http://www.pewsocialtrends.org/2014/02/11/the-rising-cost-of-not-going-to-college/}\newline
Make a comparison between the median income of college educated vs non college educated 
workers.  Remember that median income is the ``middle'' of the income bracket.  Half 
of all workers earn less than the median, and half of all workers earn more than the median.
Has the gap between the two been closing or opening?   Make a similar analysis for unemployment rates and job satisfaction.  

In addition to the article mentioned, you will need to locate other sources to explore this
general question of whether a college education really does matter to financial well being
and happiness.  Analyze the data you see in these articles and make an argument for whether 
    you believe that the expense of education is a worthwhile expense.  What factors will influence this? Perhaps now would be a good time to investigate the salary ranges in your chosen field.  Do you really need a degree to meet your career goals? Be sure to support your conclusions with a discussion of the data from the articles that you read and any analysis that you perform on this acquired data.
\end{document}
